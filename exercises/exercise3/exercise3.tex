\documentclass{article}
\usepackage[utf8]{inputenc}
\usepackage{booktabs}
\usepackage{longtable}

\title{Database Modellering og SQL Spørringer}
\author{Ditt Navn}
\date{\today}

\begin{document}

\maketitle

\section{Modellering av Flyplassdatabasesystem}

\subsection{Datamodell}
Vi har designet en datamodell for å representere flyplasser, rullebaner, fly og flyvninger. Modellen sikrer at vi kan holde oversikt over ankomster og avganger, samt unngå at to fly befinner seg på samme rullebane til samme tid.

\subsection{Tabeller}
Datamodellen er realisert gjennom følgende tabeller:

\begin{itemize}
    \item \textbf{Flyplass}: Lagrer informasjon om hver flyplass, inkludert en unik identifikasjonskode, navn og hovedtelefonnummer.
    \item \textbf{Rullebane}: Inneholder detaljer om rullebanene, hver med en unik kode, posisjon, lengde, og tilhørende flyplasskode.
    \item \textbf{Fly}: Informasjon om hvert fly, inkludert identifikasjonskode, flyselskap og antall passasjerplasser.
    \item \textbf{Flyvning}: Registrerer hver ankomst og avreise, med detaljer som flykode, rullebanekode, dato, klokkeslett, reisetype, og antall passasjerer.
\end{itemize}

\subsection{SQL Statements}
Her er noen eksempler på SQL-spørringer for å hente ut informasjon fra databasen:

\begin{verbatim}
-- Liste flyplasskode, navn, og telefonnummer
SELECT FPKode, FNavn, HovedTelefon FROM Flyplass;

-- Detaljer for rullebaner på Gardermoen (OSL)
SELECT RKode, Pos, Lengde FROM Rullebane WHERE FPKode = 'OSL';

-- Antall passasjerer som ankommer en flyplass den 19.05.05
SELECT FPKode, SUM(AntPerson) AS TotaltAntallPassasjerer 
FROM Flyvning 
JOIN Rullebane ON Flyvning.RKode = Rullebane.RKode 
WHERE Dato = '2005-05-19' AND Reise = 'Ank' 
GROUP BY FPKode;
\end{verbatim}

\section{Arbeid med Sakila-databasen}

\subsection{SELECT Statements}
Ettersom Sakila-databasens skjema er omfattende, vil denne seksjonen illustrere generelle tilnærminger for å skrive SELECT-spørringer for å hente ut relevant informasjon.

\subsection{Avanserte SQL-spørringer}
Dette inkluderer bruk av \texttt{GROUP BY}, \texttt{ORDER BY}, samt oppdatering og innsetting av data.

\section{Konklusjon}
Denne rapporten har beskrevet utformingen av en database for å administrere flyplassoperasjoner og hvordan man kan interagere med Sakila-databasen ved hjelp av SQL. Videre arbeid kan inkludere mer detaljerte spørringer for å analysere dataene på forskjellige måter.

\end{document}
