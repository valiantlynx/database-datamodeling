\documentclass{article}
\usepackage[utf8]{inputenc}
\usepackage[T1]{fontenc}
\usepackage{listings}
\usepackage{xcolor}

\lstset{
  basicstyle=\ttfamily\small,
  breaklines=true,
  breakatwhitespace=true,
  postbreak=\mbox{\textcolor{red}{$\hookrightarrow$}\space},
  language=SQL,
  showspaces=false,
  showstringspaces=false,
  frame=single,
  framesep=2pt,
  framerule=0pt,
  xleftmargin=2pt,
  xrightmargin=2pt,
  columns=flexible,
  tabsize=2
}


\title{Databaser og Datamodellering Innlevering}
\author{Gormery K. Wanjiru}
\date{\today}

\begin{document}

\maketitle

\section*{SQL Queries}

\subsection*{1a) List ut etternavn, fornavn og ansettelses-dato til samtlige selgere sortert på etternavn, deretter fornavn}

\begin{lstlisting}[language=SQL]
SELECT ENavn, FNavn, AnsDato
FROM Selger
ORDER BY ENavn, FNavn;
\end{lstlisting}

\subsection*{1b) List ut høyeste ordre-beløp som er knyttet til hver enkelt selger, forutsatt at beløpet er større enn 5000}

\begin{lstlisting}[language=SQL]
SELECT SNr, MAX(Salg)
FROM Selger
JOIN Ordre ON Selger.SNrID = Ordre.SNr
GROUP BY SNr
HAVING MAX(Salg) > 5000;
\end{lstlisting}

\subsection*{1c) Oppdater den recorden i tabellen Vare hvor MfrID = 'ACI' og PrID = '4100Y' ved å endre Pris til 3000}

\begin{lstlisting}[language=SQL]
UPDATE Vare
SET Pris = 3000
WHERE MfrID = 'ACI' AND PrID = '4100Y';
\end{lstlisting}

\subsection*{2a) List ut alle opplysningene om hvert kurs sortert i stigende rekkefølge på vekttall}

\begin{lstlisting}[language=SQL]
SELECT *
FROM Kurs
ORDER BY Vekttall ASC;
\end{lstlisting}

\subsection*{2b) List ut alle opplysningene om hvert kurs for de kurs som har vektall 3 eller mer}

\begin{lstlisting}[language=SQL]
SELECT KursNavn, KursKodeID, Vekttall, LKode
FROM Kurs
WHERE Vekttall >= 3
ORDER BY Vekttall ASC, KursNavn ASC;
\end{lstlisting}

\subsection*{2c) List ut gjennomsnittskarakteren for hver enkelt student for alle de kurs hvor eksamen ble avlagt i 1999}

\begin{lstlisting}[language=SQL]
SELECT StudNrID, StudNavn, AVG(Karakter) AS Gjennomsnittskarakter, COUNT(KursKodeID) AS AntallKurs
FROM Eksamen
JOIN Student ON Eksamen.StudNrID = Student.StudNrID
WHERE Aar = 1999 AND Karakter >= 2.0
GROUP BY StudNrID
HAVING AVG(Karakter) >= 2.0;
\end{lstlisting}

\subsection*{3a) Legg til en ny kolonne (tlf) i selgertabellen}

\begin{lstlisting}[language=SQL]
ALTER TABLE Selger
ADD tlf VARCHAR(8) DEFAULT '11111111';
\end{lstlisting}

\subsection*{3b) Legg til en ny tabell (kategori)}

\begin{lstlisting}[language=SQL]
CREATE TABLE Kategori (
    Kategori CHAR(1) NOT NULL PRIMARY KEY,
    Navn VARCHAR(50)
);

INSERT INTO Kategori (Kategori, Navn) VALUES ('A', 'Elektronikk')
INSERT INTO Kategori (Kategori, Navn) VALUES ('B', 'Mobler')
INSERT INTO Kategori (Kategori, Navn) VALUES ('C', 'Klaer')
\end{lstlisting}

\subsection*{3c) Legg til kategori i tabellen vare og oppdater varetabellen}

\begin{lstlisting}[language=SQL]
ALTER TABLE Vare
ADD Kategori CHAR(1);

-- e.g jeg oppdaterer alle varer til kategori 'A'.
UPDATE Vare
SET Kategori = 'A';

ALTER TABLE Vare
ADD CONSTRAINT FK_VareKategori FOREIGN KEY (Kategori) REFERENCES Kategori(Kategori);
\end{lstlisting}

\end{document}

